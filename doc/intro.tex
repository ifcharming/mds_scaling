\section{Introduction}

Lack of a highly scalable and parallel metadata service is the 
Achilles heal for many cluster file systems deployments in both the HPC world 
and the Internet services world \citep{hecfsio:tr06, hpcs-io:2008}.
This is because most cluster file systems have focused on scaling the
data path, i.e. provide high bandwidth parallel I/O to files that are 
at least gigabytes in size.
But with proliferation of massively parallel applications that produce 
metadata-intensive workloads, such as large number of simultaneous file creates
\cite{PLFS} and large-scale storage management \cite{ssdm}, cluster file systems 
also need to scale metadata performance.

Numerous applications and use-cases need support for concurrent and 
high-performance metadata operations.
One such example, checkpointing, requires the metadata service to
handle large number of file creates and updates at very high speeds 
\cite{PLFS}.
Another example, storage management, produces read-intensive metadata workload
that typically scans the metadata of the entire file system to perform
administration tasks for analyzing and querying metadata \cite{issdm}.

We envision a scalable metadata service with two goals. 
The first goal -- \textit{evolution, not revolution} -- emphasizes the need for
a solution that adds new support to existing cluster file systems that lack a 
scalable metadata path.
Although much new work in cluster file systems promises a distributed
metadata service \cite{ceph:weil06, 50mfiles-in-googlefs:fikes10}, it is
infeasible to replace existing cluster file systems running in large production
environments just because their metadata path does not provide the desired
scalability or the desired functionality.
Several large cluster file system installations, such as Panasas PanFS running
at LANL \cite{cielo} and PVFS running on Argonne BG/P 
\cite{pvfs:www,bgp}, can 
benefit from a solution that provides, for instance, distributed directory support 
that does not require any modifications to the running cluster file system.
The second goal -- \textit{generality} -- promises a fully, distributed and
scalable metadata service that performs well for ingest, lookups, and scans.
 
To realize these goals, this paper makes a case for a scalable metadata service 
that is designed as a middleware layered on existing cluster file systems and 
distribute file system metadata, including the namespace tree, small 
directories and large directories, across many servers.
Our key idea is to effectively synthesize a concurrent indexing 
technique to distribute metadata with a tablular, on-disk representation of all
file system metadata. 

For distributed indexing, we re-use the concurrent, incremental, hash-based
\giga{} indexing technique \citep{giga}.
The main shortcoming of the \giga{} prototype is that the splitting
the metadata partitions for better load-balancing involves migrating the
directory entries and the associated file data\citep{giga}.
This is inefficient for HPC systems where files can be gigabytes or more in
size. Our \giga{} avoids this data migration by storing all file
system objects as symbolic links: each directory entry (the name created by the
application) has a physical pathname that points to a file in the underlying
cluster file system that stores the contents of the file.
This representation of directory entries is enabled through the use a novel
on-disk metadata representation based on a log-structure merge tree (LSM-tree)
data structure \citep{ONeil1996}.
We use the \ldb{} key-value store to implement all file system metadata, including 
files, directories, and their i-node attributes, in flat files sorted on a
unique key \citep{LevelDB}.
This organization facilitates high-speed metadata creation, lookups and scans.

Effectively integrating the \ldb-based metadata store with the distributed indexing
technique requires several optimizations including cross-server split operations 
with minimum data migration, and decoupling data and metadata paths.
To demonstrate the feasibility of our approach, we implemented a prototype middleware layer
using the FUSE file system and evaluated it on 64-node cluster. Preliminary
results show promising scalability and performance: the single-node local metadata 
store was 10X faster than modern local file systems and the distributed
middleware metadata service scaled well with a peak performance of 190,000 file creates per second 
on a 64-server configuration. 


