\section{Future challenges}
\label{futurework}

Before our work can be useful in real HPC deployments, we need to address
several challenges.

First, we plan to layer our middleware on top of a real cluster file system
such as Panasas PanFS \citep{panfs:welch08} or PVFS \citep{pvfs:www}. This will
allow us to inherit the data path scalability when accessing the flat SSTable
files of \ldb that are stored on the data servers. We also plan to explore how
we can effectively leverage the fault tolerance mechanisms and system
configuration tools already present in the cluster file systems.

Second, we would like to minimize the FUSE overheads of accessing the file.
Even after the application gets a symbolic link pointing to the physical
location of the file, our current prototype will rely of FUSE and VFS to
dereference the symbolic link. Ideally, we would like to avoid this FUSE
interposition by changing the FUSE kernel module to support distributed file
system file handles. 

Third, we are exploring several optimizations for partition splitting.
Current partition splitting policy splits the affected partitions immediately.
It is possible to delay the splitting work, and use later compactions procedure
to split the affected partition. How to implement the delayed splitting policy
and make LevelDB support for links remains for future work. 

\section{Summary}
\label{summary}

Modern cluster file systems provide highly scalable I/O bandwidth along the
data path by enabling highly parallel access to file data.
Unfortunately metadata scaling is lacking behind data scaling; we propose an
idea that inherits the scalable data bandwidth of existing cluster file systems
and adds support for distributed and high-performance metadata operations.
Our idea is to integrate a distributed indexing mechanism with general-purpose
optimized on-disk metadata store.
Early prototype and evaluation shows that our approach outperforms most Linux
local file systems and scales well for large number of file creations. 

%inhertts data bandwidth and adds the metadata bandwidth scaling

%split and migrate without moving the underlying objects


