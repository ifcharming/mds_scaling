%\section*{\giga{} indexing approach}
%\label{indexing}

\giga{} is a distributed hash-based indexing technique that incrementally
divides each directory into multiple partitions that are spread over multiple 
servers \citep{giga}.
Each filename stored in a directory entry is hashed and mapped to a partition. 
using an index. 
\giga{} assumes that for any distribution of unique filenames, the hash values of 
these filenames must be uniformly distributed in the hash space.
In addition to load-balanced distribution, \giga{} also grows the directory
index incrementally, i.e. all directories start small on a single server, and
then expands to more servers as they grow in size. 

The core idea behind \giga{} is parallel splitting: each server splits 
without system-wide serialization or synchronization.
Every server makes a local decision, without coordinating with other servers, 
about when to split a partition. 
Such uncoordinated growth causes \giga{} servers to have a partial view of the
entire index; there is no central server that holds the global view of the 
partition-to-server mapping.
Each server knows about the partition it stores and the 
identity of another server that knows more about each ``child'' partition resulting
from a prior split by this server. 
This information that each server knows is called the split history of its
partitions.
The full \giga{} index is a transitive closure of the split history on each
server and represents the lineage of directory partitioning.

This full index is also not maintained synchronously by any client.
\giga{} clients can enumerate the partitions of a directory by traversing 
its split histories starting with the first partition that was created during
\texttt{mkdir()}.
However, such a full index constructed and cached by a client may be stale at
any time, particularly for rapidly mutating directories.
\giga{} allows clients to keep using the stale mapping information and updating
it in an on-demand manner. More discussion on the cost-benefit of using
inconsistent mapping state is not relevant to this work and can be found in
prior \giga{} literature \citep{giga}. 

%%%%%%%%%%%


\begin{comment}
\textbf{Tolerating inconsistent mapping at clients -- }
%\subsection*{Tolerating inconsistent mapping at clients}
%\label{indexing:inconsistency}
Clients seeking a specific filename find the appropriate partition by probing 
servers, possibly incorrectly, based on their cached index.
To construct this index, a client must have resolved the directory's parent
directory entry which contains a cluster-wide i-node identifying the server and
partition for the zeroth partition $P_0$.
Partition $P_0$ may be the appropriate partition for the sought filename, or it
may not because of a previous partition split that the client has not yet
learned about. 
An ``incorrectly'' addressed server detects the addressing error by recomputing 
the partition identifier by re-hashing the filename.
If this hashed filename does not belong in the partition it has,
this server sends a split history update to the client.
The client updates its cached version of the global index and 
retries the original request.

The drawback of allowing inconsistent indices is that clients may need 
additional probes before addressing requests to the correct server.
The required number of incorrect probes depends on the client request 
rate and the directory mutation rate (rate of splitting partitions).
It is conceivable that a client with an empty index may send O$(log(N_p))$ 
incorrect probes, where $N_p$ is the 
number of partitions, but \giga{}'s split history updates makes this many
incorrect probes unlikely.
Each update sends the split histories of all partitions that reside on a
given server, filling all gaps in the client index known to this server and
causing client indices to catch up quickly.
Moreover, after a directory stops splitting partitions, clients soon after will 
no longer incur any addressing errors.
%\giga{}'s eventual consistency for cached indices is different from LH*'s
%eventual consistency because the latter's idea of independent splitting (called
%pre-splitting) suffers addressing errors even when the index
%stops mutating \citep{lh*:litwin96}. 

\textbf{Key performance insights -- }
Detailed analysis of the scalability and performance of \giga{} studies several
tradeoffs \citep{giga}; the observations that are within the scope of this work
include the load-balancing and incremental growth strategy.


%%%%%%%%%%%
\subsection*{On-line server additions}
\label{indexing.reconfig}

%This section describes how \giga{} adapts to the addition of servers in a 
%running directory service.
%\footnote{Server removal (i.e., decommissioned, not 
%failed and later replaced) is not as
%important for high performance systems so we leave it to be done by user-level
%data copy tools.}

\begin{figure}[t]
\centerline{\includegraphics[scale=0.35]{../common/figures/giga-serveradd}}
\vspace{-10pt}
\caption{\small
\textbf{\giga{} server additions.}
By changing the partition-to-server mapping from round-robin on the original 
server set to sequential on the newly added servers, \giga{} can minimize the amount
of data migrated (shown by arrows indicating splits).
}
\label{fig:giga-adding}
%\vspace{10pt}
%\hrule depth 0.5pt
\end{figure}

When new servers are added to an existing configuration, the system is
immediately no longer load balanced, and it 
should re-balance itself by migrating a minimal number of directory entries
from all existing servers equally. 
Using the round-robin partition-to-server mapping, shown in Figure 
\ref{fig:giga-indexing}, a naive server addition scheme would require 
re-mapping almost all directory entries whenever a new server is added.

\giga{} avoids re-mapping all directory entries on addition of servers by 
differentiating
the partition-to-server mapping for initial directory growth from the mapping
for additional servers.
For additional servers, \giga{} does not use the round-robin partition-to-server
map (shown in Figure \ref{fig:giga-indexing}) and instead 
maps all future partitions to the new servers in a ``sequential manner''.
The benefit of round-robin mapping is faster exploitation of parallelism
when a directory is small and growing, while a sequential mapping for the tail
set of partitions does not disturb previously mapped partitions more than is 
mandatory for load balancing.
Figure \ref{fig:giga-adding} shows an example where the original configuration
has 5 servers with 3 partitions each, and partitions $P_0$ to $P_{14}$ use a 
round-robin rule (for $P_i$, server is $i$ \texttt{mod} $N$, where $N$ is 
number of servers).
After the addition of two servers, the six new partitions $P_{15}$-$P_{20}$
will be mapped to servers using the new mapping rule: $i$ \texttt{div} $M$, 
where $M$ is the number of partitions per server (e.g., 3 partitions/server).

In \giga{} even the number of servers can be stale at servers and clients. 
The arrival of a new server and its order in the global server list is declared
by the cluster file system's configuration management protocol, such as
Zookeeper for HDFS \citep{zookeeper:hunt10}, leading to each existing server
eventually noticing the new server.
Once it knows about new servers, an existing server can inspect its partitions
for those that have sufficient directory entries to warrant splitting and would
split to a newly added server.
The normal \giga{} splitting mechanism kicks in to migrate only directory
entries that belong on the new servers.
The order in which an existing server inspects partitions can be entirely
driven by client references to partitions, biasing migration in favor of active
directories.
Or based on an administrator control, it can also be driven by a background 
traversal of a list of partitions whose size exceeds the splitting threshold. 

\end{comment}
