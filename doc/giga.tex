%\section*{\giga{} indexing approach}
%\label{indexing}

\giga{} is a distributed hash-based indexing technique that incrementally
divides each directory into multiple partitions that are spread over multiple 
servers \cite{GIGA11}.
Each filename stored in a directory entry is hashed and mapped to a partition 
using an index. 
\giga{} selects a hash partition such that for any distribution of unique filenames, the hash values of 
these filenames will be uniformly distributed in the hash space.
In addition to load-balanced distribution, \giga{} also grows the directory
index incrementally, i.e. all directories start small on a single server, and
then expand to more servers as they grow in size. 

The core idea behind \giga{} is parallel splitting: each server splits 
without system-wide serialization or synchronization.
Every server makes a local decision, without coordinating with other servers, 
about when to split a partition. 
Such uncoordinated growth causes \giga{} servers to have a partial view of the
entire index; there is no central server that holds the global view of the 
partition-to-server mapping.
Each server knows about the partition it stores and the 
identity of another server that knows more about each ``child'' partition resulting
from a prior split by this server. 
This information is known as the per-server split history of its partitions.
The full \giga{} index is a transitive closure of the split history on each
server and represents the lineage of directory partitioning.

The full index (and split history) is also not maintained synchronously by any client.
\giga{} clients can enumerate the partitions of a directory by traversing 
its split histories starting with the first partition that was created during
\texttt{mkdir}.
However, such a full index that is cached by a client may be stale at
any time, particularly for rapidly mutating directories.
\giga{} allows clients to keep using the stale mapping information and
receiving mapping updates from servers. 
More discussion on the cost-benefit of using
inconsistent mapping state is not relevant to this work and can be found in
prior \giga{} literature \cite{GIGA11, GIGA07}. 

%%%%%%%%%%%


